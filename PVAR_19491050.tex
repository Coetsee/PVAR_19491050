\documentclass[11pt,preprint, authoryear]{elsarticle}

\usepackage{lmodern}
%%%% My spacing
\usepackage{setspace}
\setstretch{1.2}
\DeclareMathSizes{12}{14}{10}{10}

% Wrap around which gives all figures included the [H] command, or places it "here". This can be tedious to code in Rmarkdown.
\usepackage{float}
\let\origfigure\figure
\let\endorigfigure\endfigure
\renewenvironment{figure}[1][2] {
    \expandafter\origfigure\expandafter[H]
} {
    \endorigfigure
}

\let\origtable\table
\let\endorigtable\endtable
\renewenvironment{table}[1][2] {
    \expandafter\origtable\expandafter[H]
} {
    \endorigtable
}


\usepackage{ifxetex,ifluatex}
\usepackage{fixltx2e} % provides \textsubscript
\ifnum 0\ifxetex 1\fi\ifluatex 1\fi=0 % if pdftex
  \usepackage[T1]{fontenc}
  \usepackage[utf8]{inputenc}
\else % if luatex or xelatex
  \ifxetex
    \usepackage{mathspec}
    \usepackage{xltxtra,xunicode}
  \else
    \usepackage{fontspec}
  \fi
  \defaultfontfeatures{Mapping=tex-text,Scale=MatchLowercase}
  \newcommand{\euro}{€}
\fi

\usepackage{amssymb, amsmath, amsthm, amsfonts}

\def\bibsection{\section*{References}} %%% Make "References" appear before bibliography


\usepackage[round]{natbib}

\usepackage{longtable}
\usepackage[margin=2.3cm,bottom=2cm,top=2.5cm, includefoot]{geometry}
\usepackage{fancyhdr}
\usepackage[bottom, hang, flushmargin]{footmisc}
\usepackage{graphicx}
\numberwithin{equation}{section}
\numberwithin{figure}{section}
\numberwithin{table}{section}
\setlength{\parindent}{0cm}
\setlength{\parskip}{1.3ex plus 0.5ex minus 0.3ex}
\usepackage{textcomp}
\renewcommand{\headrulewidth}{0.2pt}
\renewcommand{\footrulewidth}{0.3pt}

\usepackage{array}
\newcolumntype{x}[1]{>{\centering\arraybackslash\hspace{0pt}}p{#1}}

%%%%  Remove the "preprint submitted to" part. Don't worry about this either, it just looks better without it:
\makeatletter
\def\ps@pprintTitle{%
  \let\@oddhead\@empty
  \let\@evenhead\@empty
  \let\@oddfoot\@empty
  \let\@evenfoot\@oddfoot
}
\makeatother

 \def\tightlist{} % This allows for subbullets!

\usepackage{hyperref}
\hypersetup{breaklinks=true,
            bookmarks=true,
            colorlinks=true,
            citecolor=blue,
            urlcolor=blue,
            linkcolor=blue,
            pdfborder={0 0 0}}


% The following packages allow huxtable to work:
\usepackage{siunitx}
\usepackage{multirow}
\usepackage{hhline}
\usepackage{calc}
\usepackage{tabularx}
\usepackage{booktabs}
\usepackage{caption}


\newenvironment{columns}[1][]{}{}

\newenvironment{column}[1]{\begin{minipage}{#1}\ignorespaces}{%
\end{minipage}
\ifhmode\unskip\fi
\aftergroup\useignorespacesandallpars}

\def\useignorespacesandallpars#1\ignorespaces\fi{%
#1\fi\ignorespacesandallpars}

\makeatletter
\def\ignorespacesandallpars{%
  \@ifnextchar\par
    {\expandafter\ignorespacesandallpars\@gobble}%
    {}%
}
\makeatother

\newlength{\cslhangindent}
\setlength{\cslhangindent}{1.5em}
\newenvironment{CSLReferences}%
  {\setlength{\parindent}{0pt}%
  \everypar{\setlength{\hangindent}{\cslhangindent}}\ignorespaces}%
  {\par}


\urlstyle{same}  % don't use monospace font for urls
\setlength{\parindent}{0pt}
\setlength{\parskip}{6pt plus 2pt minus 1pt}
\setlength{\emergencystretch}{3em}  % prevent overfull lines
\setcounter{secnumdepth}{5}

%%% Use protect on footnotes to avoid problems with footnotes in titles
\let\rmarkdownfootnote\footnote%
\def\footnote{\protect\rmarkdownfootnote}
\IfFileExists{upquote.sty}{\usepackage{upquote}}{}

%%% Include extra packages specified by user

%%% Hard setting column skips for reports - this ensures greater consistency and control over the length settings in the document.
%% page layout
%% paragraphs
\setlength{\baselineskip}{12pt plus 0pt minus 0pt}
\setlength{\parskip}{12pt plus 0pt minus 0pt}
\setlength{\parindent}{0pt plus 0pt minus 0pt}
%% floats
\setlength{\floatsep}{12pt plus 0 pt minus 0pt}
\setlength{\textfloatsep}{20pt plus 0pt minus 0pt}
\setlength{\intextsep}{14pt plus 0pt minus 0pt}
\setlength{\dbltextfloatsep}{20pt plus 0pt minus 0pt}
\setlength{\dblfloatsep}{14pt plus 0pt minus 0pt}
%% maths
\setlength{\abovedisplayskip}{12pt plus 0pt minus 0pt}
\setlength{\belowdisplayskip}{12pt plus 0pt minus 0pt}
%% lists
\setlength{\topsep}{10pt plus 0pt minus 0pt}
\setlength{\partopsep}{3pt plus 0pt minus 0pt}
\setlength{\itemsep}{5pt plus 0pt minus 0pt}
\setlength{\labelsep}{8mm plus 0mm minus 0mm}
\setlength{\parsep}{\the\parskip}
\setlength{\listparindent}{\the\parindent}
%% verbatim
\setlength{\fboxsep}{5pt plus 0pt minus 0pt}



\begin{document}



\begin{frontmatter}  %

\title{Revisiting `Income inequality and economic growth: a panel VAR approach'}

% Set to FALSE if wanting to remove title (for submission)




\author[Add1]{Johannes Coetsee - 19491050}
\ead{19491050@sun.ac.za}





\address[Add1]{Stellenbosch University}


\begin{abstract}
\small{
This paper attempts to replicate and critique Atems \& Jones
(\protect\hyperlink{ref-atems}{2015}), in which the authors attempt to
model the contemporaneous effects of income inequality and economic
growth of the United States at the state level, using a Panel Vector
Autoregression (PVAR) approach.
}
\end{abstract}

\vspace{1cm}

\begin{keyword}
\footnotesize{
Panel vector autoregression \sep Income Inequality \sep Economic Growth \\ \vspace{0.3cm}
\textit{} 
}
\end{keyword}
\vspace{0.5cm}
\end{frontmatter}



%________________________
% Header and Footers
%%%%%%%%%%%%%%%%%%%%%%%%%%%%%%%%%
\pagestyle{fancy}
\chead{}
\rhead{June 2021 - Econometrics 871}
\lfoot{}
\rfoot{\footnotesize Page \thepage}
\lhead{}
%\rfoot{\footnotesize Page \thepage } % "e.g. Page 2"
\cfoot{}

%\setlength\headheight{30pt}
%%%%%%%%%%%%%%%%%%%%%%%%%%%%%%%%%
%________________________

\headsep 35pt % So that header does not go over title




\hypertarget{introduction}{%
\section{\texorpdfstring{Introduction
\label{Introduction}}{Introduction }}\label{introduction}}

This paper attempts to replicate the paper by Atems \& Jones
(\protect\hyperlink{ref-atems}{2015}). It will specifically emphasise
the methodological approach used by these authors in an attempt to
uncover whether their PVAR approach holds against checks for robustness.

The paper will be structured in the following manner: for completeness,
a overview of the primary differences between the PVAR and VAR
approaches will be given, after which the contribution of Atems \& Jones
(\protect\hyperlink{ref-atems}{2015}) will be discussed. Section
\ref{Section 2.2} looks at their methodological approach through a
critical lens, thereby also informing which robustness checks are
considered the important to include in the current analysis.

\hypertarget{section-2}{%
\section{\texorpdfstring{Section 2
\label{Section 2}}{Section 2 }}\label{section-2}}

\hypertarget{panel-vector-autoregression}{%
\subsection{\texorpdfstring{Panel Vector Autoregression
\label{Section 2.1}}{Panel Vector Autoregression }}\label{panel-vector-autoregression}}

Panel Vector Autoregressions are, as the name indicates, a variation of
the standard VAR approach applied to panel datasets. Whereas VAR's treat
all variables in a given system as endogenous, the PVAR approach also
allows for unobserved individual heterogeneity. As panel data is
comprised out of various cross-sectional units of observation - in our
case, states - a VAR approach to model interactions between endogenous
variables need to account for the fact that the underlying structure
might differ across these units. A PVAR approach therefore imposes an
additional restriction, namely, that the underlying structure is the
same for all of the units of analysis. However, Love \& Zicchino
(\protect\hyperlink{ref-love}{2006}) notes that this restriction is
highly likely to be violated in practice. In order to overcome this
additional restriction on the parameter, Love \& Zicchino
(\protect\hyperlink{ref-love}{2006}) suggest the introduction of fixed
effects, thereby allowing for individual heterogeneity in the levels of
the variables. Crucially, these fixed effects are correlated with the
regressors because of the necessary inclusion of dependent variable lags
in the model (the `autoregressive' aspect of VARs), meaning that
mean-differencing - the standard method used to eliminate fixed effects
- will bias the regression coefficients. To solve this problem, Arellano
\& Bover (\protect\hyperlink{ref-arellano}{1995}) advocate for the usage
of the `Helmert procedure', where the means of only the future
observations for each unit is removed. This procedure therefore
transforms the variables in a way that preserves the orthogonality
between the variables and the lagged regressors, which, in turn, allows
for the usage of the lagged regressors as instruments whereby the
coefficients of the systems can be estimated. Moreover, these orthogonal
relationships provide the necessary moment conditions that allow for VAR
estimation using Generalized Method of Moments (GMM).

\hypertarget{inequality-and-economic-growth-atems}{%
\subsection{\texorpdfstring{Inequality and Economic Growth (Atems \&
Jones, \protect\hyperlink{ref-atems}{2015})
\label{Section 2.2}}{Inequality and Economic Growth (Atems \& Jones, 2015) }}\label{inequality-and-economic-growth-atems}}

a new comprehensive panel of annual U.S. state-level income inequality
data from 1930 to 2005 constructed by Frank (2009b), this study
reconsiders the relationship between inequality and per capita income
using a panel VAR approach. A panel VAR allows not only for the
examination of the correlation between income inequality and per capita
income, but also the dynamic responses of these variables to income and
inequality shocks, as well.

\hypertarget{methodology}{%
\subsubsection*{Methodology}\label{methodology}}
\addcontentsline{toc}{subsubsection}{Methodology}

\hypertarget{data}{%
\subsubsection*{Data}\label{data}}
\addcontentsline{toc}{subsubsection}{Data}

There are three datasets used by Atems \& Jones
(\protect\hyperlink{ref-atems}{2015}) that are relevant for our
discussion. The first, data on state-level economic growth, is measured
by the annual change in per capita real income for the 48 contiguous US
states (listed in the Appendix) for the period 1930-2005.

Atems \& Jones (\protect\hyperlink{ref-atems}{2015}) uses a panel of
state-level income inequality data (sourced by Frank
(\protect\hyperlink{ref-frank}{2009})), for the period 1930-2005 for the
United States. Their measure of inequality, the Gini coefficient, is
constructed using tax filing data. The usage of tax data is often
considered problematic in that it excludes low-income earners, thereby
introducing possibly misleading results.

overview of the data as a prelude to the estimation of the structural
VAR. Our dataset consists of annual data on the percentage change in per
capita real income and various income inequality measures for the 48

\hypertarget{results}{%
\subsubsection*{Results}\label{results}}
\addcontentsline{toc}{subsubsection}{Results}

\hypertarget{section-3}{%
\section{\texorpdfstring{Section 3
\label{Section 3 }}{Section 3 }}\label{section-3}}

\hypertarget{conclusion}{%
\section{Conclusion}\label{conclusion}}

\newpage

\hypertarget{references}{%
\section*{References}\label{references}}
\addcontentsline{toc}{section}{References}

\hypertarget{refs}{}
\leavevmode\hypertarget{ref-arellano}{}%
Arellano, M. \& Bover, O. 1995. Another look at the instrumental
variable estimation of error-components models. \emph{Journal of
econometrics}. 68(1):29--51.

\leavevmode\hypertarget{ref-atems}{}%
Atems, B. \& Jones, J. 2015. Income inequality and economic growth: A
panel var approach. \emph{Empirical Economics}. 48(4):1541--1561.

\leavevmode\hypertarget{ref-frank}{}%
Frank, M.W. 2009. Inequality and growth in the united states: Evidence
from a new state-level panel of income inequality measures.
\emph{Economic Inquiry}. 47(1):55--68.

\leavevmode\hypertarget{ref-love}{}%
Love, I. \& Zicchino, L. 2006. Financial development and dynamic
investment behavior: Evidence from panel var. \emph{The Quarterly Review
of Economics and Finance}. 46(2):190--210.

\hypertarget{appendix}{%
\section*{Appendix}\label{appendix}}
\addcontentsline{toc}{section}{Appendix}

\begin{itemize}
\tightlist
\item
  Table of 48 states and the state sample splits
\end{itemize}

\bibliography{Tex/ref}





\end{document}
